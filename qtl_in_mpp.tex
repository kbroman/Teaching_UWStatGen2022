\documentclass[aspectratio=169,12pt,t]{beamer}
\usepackage{graphicx}
\setbeameroption{hide notes}
\setbeamertemplate{note page}[plain]
\usepackage{listings}
\usepackage{eepic}

\input{header.tex}

%%%%%%%%%%%%%%%%%%%%%%%%%%%%%%%%%%%%%%%%%%%%%%%%%%%%%%%%%%%%%%%%%%%%%%
% end of header
%%%%%%%%%%%%%%%%%%%%%%%%%%%%%%%%%%%%%%%%%%%%%%%%%%%%%%%%%%%%%%%%%%%%%%

% title info
\title{QTL mapping in \\ multi-parent populations}
\author{\vspace*{-10pt} \href{https://kbroman.org}{Karl Broman}}
\institute{Biostatistics \& Medical Informatics, UW{\textendash}Madison}
\date{\href{https://twitter.com/kwbroman}{\tt \scriptsize \color{foreground} @kwbroman}
\\[-4pt]
\href{https://kbroman.org}{\tt \scriptsize \color{foreground} kbroman.org}
\\[-4pt]
\href{https://github.com/kbroman}{\tt \scriptsize \color{foreground} github.com/kbroman}
\\[-4pt]
{\scriptsize \href{https://kbroman.org/Teaching_UWStatGen2022}{\tt kbroman.org/Teaching\_UWStatGen2022}}
}


\begin{document}

% title slide
{
\setbeamertemplate{footline}{} % no page number here
\frame{
  \titlepage

  \vfill \hfill
  \href{https://creativecommons.org/publicdomain/zero/1.0/}{\includegraphics[height=7mm]{Figs/cc-zero.png}}

} }


\begin{frame}[c]{}

\vspace*{-1mm} \hspace*{-2mm}
\figw{Figs/inbredmice.jpg}{1.2}

\end{frame}



\begin{frame}[c]{}

  \figh{Figs/cc_founders.png}{1.05}

\end{frame}



\begin{frame}{}

\vspace*{13mm}

\centerline{
\begin{minipage}[t]{50mm}
\includegraphics[height=60mm]{Figs/da-vinci-man.jpg}
\end{minipage}
\hspace{15mm}
\begin{minipage}[t]{50mm}
\includegraphics[height=60mm]{Figs/vitruvian_mouse.jpg}
\hspace{5mm}
\href{http://daviddeen.com}{\scriptsize \lolit \tt daviddeen.com}
\end{minipage}
}
\end{frame}



\begin{frame}[c]{Intercross}
\figw{Figs/intercross.pdf}{1.0}
\end{frame}


\begin{frame}[c]{Data}
\figw{Figs/data_fig.png}{1.0}
\end{frame}



\begin{frame}[c]{Genome scan}
\bigskip

\only<1>{\figw{Figs/lodcurve_insulin.pdf}{1.0}}

\only<2>{\figw{Figs/lodcurve_insulin_effects.pdf}{1.0}}

\end{frame}



\begin{frame}[c]{Permutation test}
\figw{Figs/permtest.pdf}{1.0}
\end{frame}


\begin{frame}[c]{Permutation distribution}
\figh{Figs/perm_hist.pdf}{0.9}
\end{frame}



\begin{frame}[c]{QTL intervals}
\figw{Figs/lodsupint.pdf}{1.0}
\end{frame}



\begin{frame}{Multiple QTL models}

  \bbi
\item Reduce residual variation $\longrightarrow$ greater power
\item Separate linked QTL
\item Identify interactions {\lolit (epistasis)}
  \ei

\end{frame}


\begin{frame}[c]{Congenic line}
\figw{Figs/congenic.pdf}{1.0}
\end{frame}

\begin{frame}[c]{Advanced intercross lines}
\figw{Figs/ail.pdf}{1.0}
\end{frame}


\begin{frame}[c]{Recombinant inbred lines}
\figw{Figs/risib.pdf}{1.0}
\end{frame}

\begin{frame}[c]{Recombinant inbred lines (selfing)}
\figw{Figs/riself.pdf}{1.0}
\end{frame}

\begin{frame}[c]{Collaborative Cross}
\figw{Figs/ri8.pdf}{1.0}
\end{frame}

\begin{frame}[c]{MAGIC lines}
\figw{Figs/ri8self.pdf}{1.0}
\end{frame}

\begin{frame}[c]{Heterogeneous Stock/Diversity outbreds}
\figw{Figs/hs.pdf}{1.0}
\end{frame}

\begin{frame}[c]{DO genome}
\figw{Figs/do_genome.pdf}{1.0}
\end{frame}


\begin{frame}{QTL analysis in DO}

  \bbi
\item Genotype reconstruction
\item Treatment of QTL genotype
\item Kinship correction
  \ei

\end{frame}




\begin{frame}{Genotype reconstruction}
\figw{Figs/genome_reconstr.pdf}{1.0}
\end{frame}



\begin{frame}{Hidden Markov model}

\figw{Figs/hmm.pdf}{1.0}

\bigskip

{
\centering
\renewcommand{\arraystretch}{2.0}

\begin{tabular}{l@{\hspace{1cm}}l}
Initial    & $\pi(g) = \text{Pr}(G_1 = g)$ \\
Transition & $t_i(g,g') = \text{Pr}(G_{i+1} = g' \ | \ G_i = g)$ \\
Emission   & $e_i(g) = \text{Pr}(O_i \ | \ G_i = g)$
\end{tabular}

}

\end{frame}


\begin{frame}{Treatment of QTL genotypes}
  \only<1>{\figw{Figs/qtl_scan.png}{1.0}}
  \only<2>{\figw{Figs/qtl_scan_full.png}{1.0}}
\end{frame}


\begin{frame}{QTL effects (chr 1)}
\figw{Figs/qtl_effects.pdf}{1.0}
\end{frame}


\begin{frame}{QTL effects (chr 7)}
\figw{Figs/qtl_effects_2.pdf}{1.0}
\end{frame}


\begin{frame}{Linear mixed model}

\end{frame}



\begin{frame}{Research topics}

  \bbi
\item Identifying the causal polymorphisms
\item Joint analysis of high-dimensional outcomes
\item Use of intermediate biochemical traits
\item Cross-species analyses
\item QTL $\times$ QTL interactions {\lolit (epistasis)}
\item QTL $\times$ covariate interactions {\lolit (e.g. sex, diet, or environment)}
  \ei

\end{frame}



\begin{frame}[c]{}

\Large

\href{https://kbroman.org/Teaching_UWStatGen2022}{\tt kbroman.org/Teaching\_UWStatGen2022}

\vspace*{-7mm}
\hfill
\href{https://creativecommons.org/publicdomain/zero/1.0/}{\includegraphics[height=7mm]{Figs/cc-zero.png}}

\vspace{3mm}

\href{https://kbroman.org}{\tt \lolit kbroman.org}

\vspace{4mm}

\href{https://github.com/kbroman}{\tt \lolit github.com/kbroman}

\vspace{4mm}

\href{https://twitter.com/kwbroman}{\tt \lolit @kwbroman}

\vspace{4mm}

\href{https://kbroman.org/qtl2}{\tt \lolit kbroman.org/qtl2}





\end{frame}






\end{document}
